\documentclass{td_um}
\makeatletter
%--------------------------------------------------------------------------------

%\usepackage[french]{babel}
\usepackage[a4paper,hmargin=20mm,vmargin=25mm]{geometry}
\usepackage{dsfont}
\usepackage[utf8]{inputenc}
\usepackage[T1]{fontenc}
\usepackage{lmodern}

%-------------------------------- AMS --------------------------------------%
\usepackage{amsmath}
\usepackage{amsbsy}
\usepackage{amsfonts}
\usepackage{amssymb}
\usepackage{amscd}
\usepackage{amsthm}


\theoremstyle{definition}
%\newtheorem{defi}{Definition}[section]
\newtheorem{defi}{Definition}[]
\newtheorem{remark}[defi]{Remarque}
\newtheorem{prop}[]{Proposition}
\newtheorem{theo}[]{Th\'eor\`eme}



\usepackage{xcolor}
\usepackage{mathtools}
\usepackage{eurosym}
\usepackage{nicefrac}
\usepackage{latexsym}

\usepackage{multicol}
\usepackage[inline]{enumitem}
\setlist{nosep}
\setlist[itemize,1]{,label=$-$}

\usepackage{sectsty}
%\sectionfont{}
\allsectionsfont{\normalfont\sffamily\bfseries\normalsize}

\usepackage{graphicx}
\usepackage{tikz}
\usetikzlibrary{calc,fadings,decorations.pathreplacing,matrix,arrows,decorations.text}
 \usetikzlibrary{patterns}

\newcommand\chideux[1]{#1<=0||(#1!=int(#1))?1/0:x<=0?0.0:exp((0.5*#1-1.0)*log(x)-0.5*x-lgamma(0.5*#1)-#1*0.5*log(2))}
\newcommand\gauss[2]{1/(#2*sqrt(2*pi))*exp(-((x-#1)^2)/(2*#2^2))} 
\newcommand\student[1]{gamma(.5*(#1+1))/(sqrt(#1*pi)*gamma(.5*#1))*((1+x^2/#1)^(-.5*(#1+1)))}

\usepackage[framemethod=tikz]{mdframed}
\mdfdefinestyle{commandline}{leftmargin=.07\linewidth, rightmargin=.07\linewidth,backgroundcolor=gray!20,linewidth=0pt}
\mdfdefinestyle{codeline}{leftmargin=.01\linewidth, rightmargin=.01\linewidth,linewidth=1pt,backgroundcolor=gray!20,linewidth=0pt}


\usepackage{textcomp}
\usepackage{listings}
\lstloadlanguages{R}
\lstset{upquote=true,
	%columns=flexible,
	keepspaces=true,
	breaklines=true,
	breakindent=0pt,
	basicstyle={\small\ttfamily}
	}
\lstset{extendedchars=true,
	literate={é}{{\'e}}1
		 {è}{{\`e}}1
		 {à}{{\`a}}1
		 {ç}{{\c{c}}}1
		 {œ}{{\oe}}1
		 {ù}{{\`u}}1
		 {É}{{\'E}}1
		 {È}{{\`E}}1
		 {À}{{\`A}}1
		 {Ç}{{\c{C}}}1
		 {Œ}{{\OE}}1
		 {Ê}{{\^E}}1
		 {ê}{{\^e}}1
		 {î}{{\^i}}1
		 {ô}{{\^o}}1
		 {û}{{\^u}}1
		 {°}{{\degre}}1
	}

\lstnewenvironment{script}{%
\lstset{language=,
	    aboveskip=0pt,
	belowskip=0pt
	}
}{}
\surroundwithmdframed[style=commandline]{script}

\lstdefinestyle{codeR}{%
	aboveskip=0pt,
	belowskip=0pt,
	%frame=single,
	commentstyle=\color{green!50!black}\itshape,
	language=R,
	keywordstyle=\color{violet!75},
	stringstyle=\color{red},
	showstringspaces=false,
      %  emph={output,input}, emphstyle=\color{brown!75},
      %  emph={maFonction}, emphstyle=\underline,
	}

\lstnewenvironment{codeR}[1][]{\lstset{style=codeR,#1}}{}
\surroundwithmdframed[style=codeline]{codeR}


\usepackage{algorithm}
\usepackage{algpseudocode}


\usepackage{pgfplots}
\usepgfplotslibrary{fillbetween}
\pgfplotsset{compat=newest}
%\usepgfplotslibrary{external} 
%\tikzexternalize[prefix=./output_latex/]
%\DeclareSymbolFont{RalphSmithFonts}{U}{rsfs}{m}{n}
%\DeclareSymbolFontAlphabet{\mathscr}{RalphSmithFonts}
%\def\mathcal#1{{\mathscr #1}}



% ------------------------------------------- Command -------------------------------%
\providecommand{\abs}[1]{\left|#1\right|}
\providecommand{\norm}[1]{\left\Vert#1\right\Vert}
\providecommand{\U}{\mathcal{U}}
\providecommand{\R}{\mathbb{R}}
\providecommand{\Cc}{\mathcal{C}}
\providecommand{\reg}[1]{\mathcal{C}^{#1}}
\providecommand{\1}{\mathds{1}}
\providecommand{\N}{\mathbb{N}}
\providecommand{\Z}{\mathbb{Z}}
\providecommand{\C}{\mathbb{C}}
\providecommand{\F}{\mathbb{F}}
\providecommand{\K}{\mathbb{K}}
\providecommand{\p}{\partial}
\providecommand{\gR}{\textsc{R}}
\providecommand{\one}{\mathds{1}}
%\renewcommand{\E}{\mathbb{E}}
\renewcommand{\P}{\mathbb{P}}
\providecommand{\ie}{\textit{i.e. }}

\newcommand\rst[2]{{#1}_{\restriction_{#2}}}
%Operateur
\providecommand{\abs}[1]{\left\lvert#1\right\rvert}
\providecommand{\sabs}[1]{\lvert#1\rvert}
\providecommand{\babs}[1]{\bigg\lvert#1\bigg\rvert}
\providecommand{\norm}[1]{\left\lVert#1\right\rVert}
\providecommand{\bnorm}[1]{\bigg\lVert#1\bigg\rVert}
\providecommand{\snorm}[1]{\lVert#1\rVert}
\providecommand{\prs}[1]{\left\langle #1\right\rangle}
\providecommand{\sprs}[1]{\langle #1\rangle}
\providecommand{\bprs}[1]{\bigg\langle #1\bigg\rangle}

\DeclareMathOperator{\deet}{Det}
\DeclareMathOperator{\supp}{Supp}
\DeclareMathOperator{\sinc}{sinc}


\usepackage{ifthen}

\newcommand{\eno}[1]{%
	\ifthenelse{\equal{\version}{eno}}{#1}{}%
}
\newcommand{\cor}[1]{%
        \ifthenelse{\equal{\version}{cor}}{
\medskip 

{\small \color{gray} #1}

\medskip 
}{}
}

%------------------------------------------------------------------------------
\DeclareUnicodeCharacter{00A0}{~}
\makeatother



\def\version{eno}
%\def\version{cor}

\usepackage{hyperref}
\ue{HMMA201}

\providecommand{\T}{\mathbb{T}}
\providecommand{\1}{\mathds{1}}
\title{TD III}


\newcommand{\miniscule}{\@setfontsize\miniscule{5}{6}}
%-----------------------------------------------------------------------------
\begin{document}
\maketitle


\exo{}  Montrer que la statistique de test $F$, introduite en cours pour tester la validité d'un modèle emboîté, peut s'écrire
\[
F=\frac{n-p}{q} \frac{R^{2}-R_{0}^{2}}{1-R^{2}}
\]
où $R^{2}$ et $R_{0}^{2}$ sont les coefficients de détermination associés respectivement au modèle complet et au modèle emboîté.

%\newpage

\exo{} Dans le modèle de régression linéaire, il arrive parfois que l'on souhaite imposer des contraintes linéaires à $\beta$, par exemple que sa première coordonnée soit égale à $1$. Nous supposerons en général que nous imposons $q$ contraintes linéairement indépendantes à $\beta$, ce qui s'écrit sous la forme : $R \beta=r$, où $R$ est une matrice $q \times p$ de rang $q<p$ et $r$ un vecteur de taille $q$. Montrer que l'estimateur des moindres carrés sous contraintes s'écrit:
\[
\hat{\beta}_{c}=\hat{\beta}+\left(X^{t} X\right)^{-1} R^{t}\left[R\left(X^{t} X\right)^{-1} R^{t}\right]^{-1}(r-R \hat{\beta}).
\]
Calculer  $\mathbb{E}\left(\hat{\beta}_{c}\right)$ et  $\mathbb{V}\left(\hat{\beta}_{c}\right)$

%\newpage

\exo{Modèle de Cobb-Douglas} Nous disposons pour $n$ entreprises de la valeur du capital $K_{i},$ de l'emploi $L_{i}$ et de la valeur ajoutée $V_{i}$. Nous supposons que la fonction de production de ces entreprises est du type Cobb-Douglas:
\[
V_{i}=\lambda L_{i}^{\alpha} K_{i}^{\gamma}
\]
\begin{enumerate}
\item Comment se ramène-t-on à un modèle de régression linéaire?
\item Pour $n=1658$ entreprises, nous avons obtenu les estimateurs suivants :
\[
    \hat{\beta}=\begin{pmatrix}3,136 \\ 0,738 \\ 0,282\end{pmatrix}
\]
avec $R^{2}=0,945$ et $S C R=148,27$. Nous donnons aussi
\[
    \left(X^{t} X\right)^{-1}=\left(\begin{array}{ccc}
            0,0288 & 0,0012 & -0,0034 \\
            0,0012 & 0,0016 & -0,0010 \\
            -0,0034 & -0,0010 & 0,0009
    \end{array}\right) \text { et } X^{t} X=\left(\begin{array}{ccc}
            423 & 2231 & 4077 \\
            2231 & 13808 & 23769 \\
            4077 & 23769 & 42923
    \end{array}\right)
\]
Calculer $\hat{\sigma}^{2}$ et une estimation de $\mathbb{V}(\hat{\beta})$.
\item Donner un intervalle de confiance au niveau $95 \%$ pour $\alpha$. Idem pour $\gamma$.
\item Tester au niveau $5 \%$ l'hypothèse $H_{0}: \gamma=0,$ contre $H_{1}: \gamma>0$.
\item Nous voulons tester l'hypothèse selon laquelle les rendements d'échelle sont constants (une fonction de production $F$ est à rendement d'échelle constant si $\left.\forall \theta \in \mathbb{R}^{+}, F(\theta L, \theta K)=\theta F(L, K)\right)$. Quelles sont les contraintes verifiées par le modèle lorsque les rendements d'échelle sont constants? Tester au niveau $5 \%$ l'hypothèse  $H_{0}:$ "les rendements sont constants", contre $H_{1}:$ "les rendements sont croissants".
    \end{enumerate}


\end{document}

